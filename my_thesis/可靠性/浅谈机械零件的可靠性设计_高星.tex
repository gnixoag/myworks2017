%%%% 导言区
%% 设定纸张大小为A4, 基本字体大小为12pt, 文章题目单独为一页, 
%% 文档类型为article
\documentclass[a4paper,12pt]{article}

%% en_preamble包含基本的宏包配置
%%%%%%%%------------------------------------------------------------------------
%%%% 日常所用宏包

%% 控制页边距
\usepackage[top=2cm, bottom=2cm, left=2.cm, right=2.cm,includehead,includefoot]{geometry}

%% 控制项目列表
\usepackage{enumerate}

%% 多栏显示
\usepackage{multicol}

%% hyperref宏包,生成可定位点击的超链接,并且会生成pdf书签
\usepackage[%
    pdfstartview=FitH,%
    CJKbookmarks=true,%
    bookmarks=true,%
    bookmarksnumbered=true,%
    bookmarksopen=true,%
    colorlinks=true,%
    citecolor=blue,%
    linkcolor=blue,%
    anchorcolor=green,%
    urlcolor=blue%
]{hyperref}

\ctexset {
	section = {
		name = {理论},
		format = \bf \zihao{4} \centering
	},
	subsection = {
		name = {},
		number = \Roman{subsection},
		format = \bf \zihao{4} 
	},
	subsubsection = {
		name = {,、\hspace{-0.5em}},
		number = \chinese{subsubsection},
		format = \bf \zihao{4} 
},
paragraph = {
name = {(,)\hspace{-0.5em}},
number = \chinese{paragraph},
format = \bf \zihao{4} 
},
subparagraph = {
name = {,、\hspace{-0.5em}},
format = \bf \zihao{4}
}
}

\setcounter{secnumdepth}{5}


%% 控制目录
\usepackage{titletoc}

%% 支持彩色文本、底色、文本框等
\usepackage{color,xcolor}

%%%% 基本插图方法
%% 图形宏包
\usepackage{graphicx}

%%%% pgf/tikz绘图宏包设置
\usepackage{pgf,tikz}
\usetikzlibrary{shapes,automata,snakes,backgrounds,arrows}
\usetikzlibrary{mindmap}

%%%% fancyhdr设置页眉页脚
%% 页眉页脚宏包
\usepackage{fancyhdr}

%% 页眉页脚风格
\pagestyle{plain}

%%%% 设置listings宏包用来粘贴源代码
%% 方便粘贴源代码,部分代码高亮功能
\usepackage{listings}

%% 所要粘贴代码的编程语言
%\lstloadlanguages{}

%% 设置listings宏包的一些全局样式
%% 参考http://hi.baidu.com/shawpinlee/blog/item/9ec431cbae28e41cbe09e6e4.html
\lstset{
showstringspaces=false,              %% 设定是否显示代码之间的空格符号
numbers=left,                        %% 在左边显示行号
numberstyle=\tiny,                   %% 设定行号字体的大小
basicstyle=\footnotesize,                    %% 设定字体大小\tiny, \small, \Large等等
keywordstyle=\color{blue!70}, commentstyle=\color{red!50!green!50!blue!50},
                                     %% 关键字高亮
frame=shadowbox,                     %% 给代码加框
rulesepcolor=\color{red!20!green!20!blue!20},
escapechar=`,                        %% 中文逃逸字符,用于中英混排
xleftmargin=2em,xrightmargin=2em, aboveskip=1em,
breaklines,                          %% 这条命令可以让LaTeX自动将长的代码行换行排版
extendedchars=false                  %% 这一条命令可以解决代码跨页时,章节标题,页眉等汉字不显示的问题
}
%%%% listings宏包设置结束

%% 设定段间距
\setlength{\parskip}{0.3\baselineskip}

%% 设定行距
\linespread{1}

\usepackage{tabu} % 用tabu代替 array
\usepackage{multirow}
\usepackage{zhnumber}
\usepackage{calc,marvosym,ifthen,fancybox,url,layout}
\setcounter{tocdepth}{1}
\usepackage{paralist}

%给旁注加个黑原点
\usepackage{wasysym}
\let\marginparNR\marginpar
\def\marginpar#1{\marginparNR{\textcolor{red}{ \CIRCLE{}   #1  }}}

%调整列表前后的间距
\makeatletter
\let\orig@Enumerate=\enumerate
\renewenvironment{enumerate}{\orig@Enumerate}{\vspace{-0.5cm}\endlist}
\let\orig@Itemize=\itemize
\renewenvironment{itemize}{\orig@Itemize}{\vspace{-0.5cm}\endlist}
\makeatother

%给目录进行设定
\titlecontents{section}[0pt]{\addvspace{5pt}\filright}
{ \thecontentslabel \hspace{0.5em} }
{}{\titlerule*[8pt]{.}\contentspage}


%画边框
%\def\boxhack{\leavevmode\vbox to0pt{\vss\rlap{\hskip 320pt
%			\setlength{\unitlength}{1pt}\cornersize*{10pt}\thicklines\fancyoval(365,675)}\vskip -680pt}}
%\def\boxhackb{\leavevmode\vbox to0pt{\vss\rlap{\hskip 80pt
%			\setlength{\unitlength}{1pt}\cornersize*{10pt}\thicklines\fancyoval(100,675)}\vskip -680pt}}

%用tikz画边框	
\def\biankuang{\leavevmode\vbox to0pt{
		\vss\rlap{\hskip 0.8cm
			\tikz \draw(4,0)--(0,0)--(0,-22.5)--(17,-22.5)--(17,0)--(4,0)--(4,-22.5);		
		}\vskip -22.7cm}}

\newcolumntype{M}[1]{>{\zihao{4}\centering\arraybackslash}m{#1}}
\newcolumntype{N}{@{}m{0pt}@{}}

\newcommand{\ktmq}[1]{\gdef\ktmqNR{#1}}%课题名称
\newcommand{\jxmb}[1]{\gdef\jxmbNR{#1}}%教学目标
\newcommand{\jxnd}[1]{\gdef\jxndNR{#1}}%教学难点
\newcommand{\jxzd}[1]{\gdef\jxzdNR{#1}}%教学重点
\newcommand{\jjff}[1]{\gdef\jjffNR{#1}}%解决方法
\newcommand{\jxhj}[1]{\gdef\jxhjNR{#1}}%教学后记

\newcommand{\jc}[1]{\gdef\jcNR{#1}}%教材
\newcommand{\cks}[1]{\gdef\cksNR{#1}}%参考书
\newcommand{\jsxm}[1]{\gdef\jsxmNR{#1}}%教师姓名
\newcommand{\jyszr}[1]{\gdef\jyszrNR{#1}}%教研室主任

\newcommand{\skbc}[1]{\gdef\skbcNR{#1}}%授课班次
\newcommand{\skrq}[1]{\gdef\skrqNR{#1}}%授课日期
\newcommand{\biaoti}[1]{\gdef\biaotiNR{#1}}%标题头

\newcounter{thesectionSY}

\newcommand{\makeshouye}{
	\setcounter{thesectionSY}{\thesection+1}
	\restoregeometry	
	\renewcommand{\headrulewidth}{0pt}
	\pagestyle{fancy}
	\fancyhead{}
	\lhead{} 
	\chead{
		\begin{tabular}{@{\hspace{1.2cm}}M{7cm}@{\hspace{-0.4cm}}M{8cm}N}
			\parbox{7cm}{\linespread{0.2}
				\makebox[7cm][s]{\kaishu \zihao{3} 湖南九嶷职业技术学院}\\ 
				\makebox[7cm][s]{\kaishu \zihao{3} 湖南潇湘技师学院}
			}
			&  \makebox[6cm][s]{\rule{0pt}{0.9cm}\zihao{1} \heiti \kaishu 授课课时计划}\\
		\end{tabular}
	}
	
	\begin{tabular}{M{2.2cm}|M{7cm}|M{5.8cm}N}
		\hline
		\multirow{2}*{
			\rule{0pt}{1.4cm}\parbox[b]{2.cm}{
				\centering 课\hfill 程\hfill 章\hfill 节\\及\hfill 主\hfill 题}}& \heiti \biaotiNR\thethesectionSY  &  ~授~课~教~师\hfill {\heiti \zihao{4} \underline{\jsxmNR}}\hfill 签字~~~&\\[0.6cm] \cline{2-3}
		
		& \heiti \ktmqNR &  ~教研室主任\hfill {\fangsong \bf \zihao{4} \underline{\jyszrNR}}\hfill 签字~~~&\\[0.6cm]\hline
		
		\multicolumn{3}{l}{
			\begin{minipage}[t][3.7cm][t]{15cm}	
				\begin{minipage}[t]{2.5cm}
					\vspace{6pt} \hfill \zihao{4} 教学目标:
				\end{minipage}\hspace{0.5cm}				
				\begin{minipage}[t][4.3cm][t]{12cm}
					\vspace{0pt}\zihao{4} \setlength{\baselineskip}{12pt} 
					\begin{enumerate}[1、]
						\jxmbNR
					\end{enumerate} 
				\end{minipage} 
			\end{minipage}
		}\vspace{0.3cm} &\\ \hline
		\multicolumn{3}{l}{
			\begin{minipage}[t][4cm][t]{15cm}
				\begin{minipage}[t]{2.5cm}
					\vspace{5pt} \hfill \zihao{4} 教学重点:
				\end{minipage}\hspace{0.5cm}				
				\begin{minipage}[t]{12cm}
					\vspace{0pt} \zihao{4} \setlength{\baselineskip}{12pt} 
					\begin{enumerate}[1、] \jxzdNR \end{enumerate}
					\vspace{7pt} 
				\end{minipage}
				\vspace{5pt} 
				\begin{minipage}[t]{2.5cm}
					\vspace{6pt} \hfill \zihao{4} 教学难点:
				\end{minipage}\hspace{0.5cm}		
				\begin{minipage}[t]{12cm}
					\vspace{0pt} \zihao{4} \setlength{\baselineskip}{12pt} 
					\begin{enumerate}[1、] \jxndNR \end{enumerate}
					\vspace{0pt} 	
				\end{minipage}
				\begin{minipage}[t]{2.5cm}
					\vspace{6pt} \hfill \zihao{4} 解决方法:
				\end{minipage}\hspace{0.5cm}		
				\begin{minipage}[t]{12cm}
					\vspace{6pt}\zihao{4} \jjffNR
				\end{minipage}
				
			\end{minipage}
		} &\\  \hline
		
		\multirow{2}*{ 	\rule{0pt}{1.4cm}\parbox[b]{2.cm}{
				\centering 教\hfill 材\hfill 和\\参\hfill 考\hfill 书 } } & \multicolumn{2}{c}{\zihao{4} \jcNR } &\\[0.6cm] \cline{2-3}
		&  \multicolumn{2}{c}{\zihao{4} \cksNR } &\\[0.6cm] \hline
		\multirow{2}*{\rule{0pt}{1.4cm}\parbox[b]{2.cm}{
				\centering 授\hfill 课\hfill 班\hfill 次\\授\hfill 课\hfill 日\hfill 期 } } & \multicolumn{2}{c}{ \skbcNR } &\\[0.6cm] \cline{2-3}
		&\multicolumn{2}{c}{ \skrqNR } &\\[0.6cm] \hline
		
		\multicolumn{3}{l}{
			\begin{minipage}[t]{2.5cm}
				\vspace{0pt} \hfill \zihao{4} 教学后记:
			\end{minipage}\hspace{0.5cm}				
			\begin{minipage}[t][4.5cm][t]{12cm}
				\vspace{0pt}\zihao{4} \jxhjNR
			\end{minipage} 
		}\vspace{0.3cm} &\\ \hline	
	\end{tabular}
	\newpage
	\newgeometry{textwidth={\textwidth-150pt},top=2cm,bottom=2cm,right=2.5cm,includehead,includefoot,marginparsep=28pt,marginparwidth=85pt}
	
	\reversemarginpar
	\fancyhead{} 
	\chead{\hspace{1cm} \kaishu \zihao{1} 教 \hspace{1cm} 案 \hspace{1cm} 纸 }
	%\lhead{\boxhack \boxhackb } %边框 
	\lhead{ \biankuang}%边框 
	\zihao{4}		

	\section{ \ktmqNR }
}



%% 如果不写中文的话就不需要引用xecjk_preamble里面的配置
\input{data/xecjk_preamble}

\title{浅谈机械零件可靠性设计}
\author{高星}
%\maketitle

%%%% 导言区结束
%%%%%%%%------------------------------------------------------------------------

%%%%%%%%------------------------------------------------------------------------

%%%% 公共信息

%%%% 正文部分
\begin{document}
	
\begin{titlepage}
\begin{center}
\includegraphics[height=2.8cm]{images/hnsfdx} \vspace{1cm}

{ \bf \kai  \fontsize{42pt}{80pt}\selectfont 工程与设计学院 }
\vfill
{ \bf \hei  \fontsize{42pt}{80pt}\selectfont 研究生作业 }
\vfill \vfill

\sanhao \linespread{2}
 \makebox[5em][s]{课程名称}:~
\underline{ \makebox[7cm]{ 浅谈机械零件可靠性设计 }}\\
 \makebox[5em][s]{教师姓名}:~
\underline{ \makebox[7cm]{朱瑞林}}\\
 \makebox[5em][s]{研究生姓名}:~
\underline{ \makebox[7cm]{高星}}\\
 \makebox[5em][s]{学号}:~
\underline{ \makebox[7cm]{201580180367}}\\ \makebox[5em][s]{专业方向}:~
\underline{ \makebox[7cm]{机械工程}}\\			   
\vfill
2017年2月10日
\end{center}
\end{titlepage}

\sihao
\begin{center}
	 \erhao  \hei  浅谈机械零件可靠性设计
\end{center} 
\sihao \setlength{\parindent}{2em}    

{\bf 摘 要: }机械零件可靠性设计是保证机械产品可靠性的基础和关键。首先介绍机械零件可靠性的特点及其可靠性参数、 机械可靠性设计与传统机械设计之间的关系;结合可靠性理论对 机 械 可 靠 性 设 计 理 论 进 行 分 析 ,阐 明 了 可 靠性优化设计、可靠性灵敏度设计、可靠性稳健设计、可 靠 性 试 验、传 统 设 计 方 法 与 可 靠 性 设 计 相 结 合 等 机 械 零 件 可 靠性设计理论与方法的内涵,为机械零件可靠性设计提供方法 。

{\bf 关键词: }机械零件;可靠性设计;可靠性优化设计;可靠性灵敏度设计;可靠性稳健设计

\setcounter{section}{-1}
\section{引言} 
\sihao \setlength{ \baselineskip }{25pt}
现代机械零件质量是否合格的四大指标分别为:可靠性、安全性、经济性、性能,而可靠性是其最核心指标,并成为机械领域关注的焦点。
随着科学技术的飞速发展,对机械产品可靠性要求也不断提高。要提高机械产品的可靠性,首先应从设计上提高机械产品的固有可靠性,然后在制造中加以保证。从某种意义上讲,从事机械零件与系统的可靠性预计、评价及可靠性设计是提高机械产品可靠性的有效途径。


所谓可靠性就是指在规定时间和条件下产品完成特定功能的能力,可分为:固有可靠性、使用可靠性和环境适应性。机械系统随着科技发展的脚步变得越来越精密、越来越复杂,研究机械的可靠性就成了急需解决的问题。
%对可靠性的研究包括很多方面,如:可靠性工程技术、可靠性试验、可靠性优化等问题。本文将从多方面对可靠性的发展做出总结。


\section{机械可靠性的特点}
\subsection{机械产品可靠性预计困难}
由于机械产品的失效机理复杂多变,加之缺乏准确完整的数据,机械零部件的可靠性就很难预计;另外,由于机械产品可靠性模型难于建立,很多依赖于系统可靠性模型的预计方法也很难应用于机械产品可靠性预计中。
\subsection{机械产品的故障模式具有多样性和复杂性}
机械产品的故障模式与其材料、具体结构、载荷性质和大小等有密切关系,故障模式之间还存在着相关性。实现同一功能要求,由于采用不同的结构形式,可以改变机械产品零部件的应力状态。失去规定的功能可以是损坏、失调、渗漏、堵塞、老化、松脱或它们的组合等多种表现形式。一个零部件可能有多种故障模式,同一故障模式可能发生在不同部位,增加了故障模式分析的难度和复杂性。
\subsection{机械零部件通用化、标准化程度低}
机械产品的大多数零部件都是非标准件,只有少数零部件如轴承、密封件、阀、泵等已实现标准化、通用化。大部分零部件由于功能、结构各异,只能将其特征参数(如齿轮模数、液压缸直径、螺纹直径等)标准化。设计人员在系统设计的同时,还要根据具体结构要求及载荷性质、几何尺寸进行零部件设计。而机械可靠性设计的难点之一是缺乏材料强度和载荷分布的数据,难以给出像电子元器件那样工程上实用的机械零部件故障率手册。
\subsection{机械零部件的故障既有偶然性故障,又有耗损性故障}
后者的故障机理大多与磨损、疲劳、腐蚀、老化等耗损过程密切相关,具有渐变性的特点,故障率是时间的函数,渐变性失效是通过极限状态准则(即耐久性准则)进行判断,这与电子元器件以偶然性故障为主的特点有所不同,用故障率为常数的数学模型描述也受到局限。所以机械产品的寿命问题是主要的,有必要引入耐久性。
\section{机械产品的可靠性参数}

机械可靠性的参数较多, 如可靠度、 可靠寿命、失 效率、 平均寿命、平均故障间隔时间等。对于机械系统 而言, 大部分系统是可修复的, 从设计的角度对机械产 品不仅要求其具有先进的技术性能和良好的 使用效 果, 还要具有合理的经济性。选用哪些参数进行可靠 性设计比较合适, 应根据各产品的具体情况和用户的 要求。而对机械零件而言, 大多是不可修复的, 在维修 时一般给予更换。所以在设计机械零件时, 不但要确 保其可靠度, 更为重要的是使其设计寿命达到可靠性 要求。无论对系统和零件而言, 可靠度都是设计时最 为通用和重要的指标。

\subsection{零件的可靠度}
机械零部件设计的基本目标是在一定的可靠度下 保证其危险断面上的最小强度不低于最大的应力。应 力和强度都不是一个确定的值, 而是由若干随机变量 组成的多元随 机变量。由强 度 $S $ 的概 率密度 函数$ f (S )$和应力$ s$的概率密度函数 $f ( s)$可得零件的可靠度 $R i$:
$$  Ri = \int_{-\infty}^{+\infty} f(s)\left( \int_s^\infty f(S)dS \right) ds $$

\subsection{平均寿命、 平均故障间隔时间}
平均寿命是产品寿命的平均值, 常用于评价一批 产品的可靠性, 平均寿命对于不可修复和可修复产品, 含义分别为:
\begin{itemize}
	\item  对于不可修复的产品, 其寿命是指它失效前的 工作时间。因此, 平均寿命就是指该产品从开始使用 到失效前的工作时间的平均值, 一般称为失效前平均 时间, 记为$ MTTF $。
	\item 对可修复产品, 其平均寿命是指故障间隔时间 的平均值, 记为 $MTBF$。
\end{itemize}
\subsection{可靠寿命}
任何机械系统均有设计寿命问题, 在设计寿命范 围内, 并不是所有机械系统都能可靠工作, 为此需要用 可靠寿命来度量设计寿命内可靠工作的能力。

 可靠寿命是给定的可靠度对应的工作时间, 记为$ t (R ) $ 。
\subsection{失效率}
失效率也是机械可靠性设计常用的参数, 其定义 是工作到某时刻 $t $时尚未失效的产品, 在该时刻 t以后 的下一个单位时间发生失效的概率。

\section{传统设计和可靠性设计}
\subsection{传统设计}
传统设计方法是以直觉设计、经验设计、静态设计、为基础的设计方法、在传统的机械设计方法进行设计时、不能预测零部件在运行中破环的概率,一是因为在设计中所采用的载荷、材料性能等数据,是他们的平均值,没有考虑到数据的分散性,而是为了保证机械的可靠性,往往对计算载荷、选用强度等分别乘以各种系数,如载荷系数、尺寸系数、齿宽系数等,对于实际的应力,或者其他数值与理论推导不同时,还要有各种修正系数;最后,还要考虑到安全系数,这种传统方法是人们对这些因素的随机变化所做的经验估计,并不等于因素的实际情况,与此同时,由于对这些随机变化的因素无法进行精确计算,只好将尺寸、重量等作又不精确的放大。
\begin{enumerate}
	\item 理论设计;根据长期总结出来的设计理论或者实验数据所进行的设计,称为理论设计。
	\item 经验设计;根据对某类零件已有的设计或使用实践而归纳出经验关系式,或根据设计者本人的工作经验用类比的方法所进行的设计叫做经验设计,这对那些使用要求变动不大结构形状已经典型化的零件,是很有效的设计方法,但是这种方法是经验设计,精确度和可靠性无法定量化的保证,并且现在的机械系统越来越复杂,技术含量越来越高,经验设计的方法已经不能再满足市场需要。
	\item 模型试验设计;对于一些尺寸巨大而结构又很复杂的重要零件,尤其是一些重型整体机械零件,为了提高设计质量,可以采用模型实验设计的方法,即把初步设计的零件、部件或者机器制成小模型或小尺寸样机,经过实验的手段对其各方面的性能进行检验,根据实验结果对设计进行逐步修改达到完善,但是这种方法费时、昂贵,因此只用于特别重要的设计中。
\end{enumerate}


\subsection{可靠性设计}
20世纪初期 把 概 率 论 及 数 理 统 计 学 应 用 于 结 构安全度分析,已标 志 着 结 构 可 靠 性 理 论 研 究 的 初 步开始 .20世纪 40年代以来,机 械 可 靠 性 设 计 理 论有了长足的发展,目前为止已经相当成熟,尤其是许 多国家开始研究在 结 构 设 计 规 范 中 的 应 用,使 机 械可靠性设计理论的 应 用 进 入 一 个 新 的 时 期 .

对当前机械产品而言,如何提高设计质量、完善 设计理论、改 进 设 计 技 术、缩 短 设 计 周 期 是 最 重 要 的,而这些都与可靠性有着密切的联系 .可靠性技术已深入机械零部件 结 构 设 计、强 度 设 计 以 及 失 效 分 析中,这些研究标志 着 可 靠 性 理 论 已 经 进 入 实 用 阶 段 .机械零件可靠性 理 论 研 究 工 作 已 经 成 为 机 械 工 程中的研究热点,目前有大量论文和专著,已证实结 构系统可靠性分析 和 计 算 方 法 相 当 成 熟 .就 目 前 的
发展趋势看如下几方面应是工程机械结构可靠性理 论研究的热点 .
\begin{enumerate}
	\item 可靠性优化设计;可靠性优化设计是在可靠性基础上进行优化设 计,既能定量地满足产品在运行中的可靠性,又能使 产品的尺寸、成本、质 量、体 积 和 安 全 性 等 参 数 得 到 优化,从而保证结构 的 预 测 工 作 性 能 与 实 际 工 作 性 能更符合 .该方法将 可 靠 性 分 析 理 论 与 数 学 规 划 方 法有机地结合在一 起,在 对 各 参 数 进 行 可 靠 性 优 化 设计时,首先以机械 零 件 的 可 靠 度 作 为 优 化 的 目 标 函数,使零件的某些 指 标 (成 本、质 量、体 积 或 尺 寸) 最小化,再以强度、刚 度、稳 定 性 等 设 计 要 求 为 约 束 条件建立可靠性优 化 设 计 数 学 模 型,根 据 模 型 的 规 模、性态、复杂程度 等 因 素 选 择 合 适 的 优 化 方 法,最 后求出最优设 计 变 量 。
	\item 可靠性灵敏度设计;可靠性灵敏度设计是在可靠性基础上进行灵敏 度设计,充分反映各 设 计 参 数 对 机 械 产 品 失 效 影 响 的不同程度,便于找 出 哪 些 随 机 变 量 对 机 械 零 件 可 靠性设计的敏感性 影 响 较 大,并 对 此 参 数 进 行 重 分 析和再设计 .即通过 估 计 设 计 变 量 变 差 和 约 束 变 差 对质量性能指标影 响 的 大 小,改 变 设 计 参 数 中 影 响 较大的参数以使产品对可控因素变差和不可控因素 变差的影响失去灵 敏 性 .可 靠 性 灵 敏 度 设 计 首 先 建 立极限状态方程,然后对各设计参数求偏导数,得到 可靠性的灵敏度计 算 公 式,进 而 确 定 各 设 计 参 数 的 灵敏度,用灵敏度数 值 作 为 再 设 计 时 修 改 设 计 参 数 的依据,从而使得 参 数 修 改、再 设 计 工 作 事 半 功 倍 。
	\item 可靠性稳健设计;可靠性稳健设计是在可靠性设计、优化设计、灵 敏度设计和稳健设计的基础上 进 行 的 ;稳 健 设 计 能使产品的性能对在制造期间的变异或使用环境的 变异不敏感,并 使 产 品 在 其 寿 命 周 期 内,不 管 其 参数、结构发生漂移或老化(小范围内),都能持续可靠 地工作的一种设计 方 法 .可 靠 性 稳 健 设 计 在 设 计 阶 段通过灵敏度分析,使产品在不消除、不减少不确定 性因素的情况下,通 过 设 计 使 不 确 定 性 因 素 对 产 品 质量影响的敏感程度最小,从而提高产品质量、降低 产品成本 。
	\item 可靠性试验;就 目 前 来 看 可 靠 性 理 论 研 究 可 以 说 相 当 成 熟, 但是可靠性试验却 不 是 很 完 善 .可 靠 性 试 验 是 对 产 品的可靠性进行调研、分析和评价的一种手段,其目 的是发现产品在设计、材料和工艺方面的各种缺陷, 为改善产品的战备完好性、提高任务成功率、减少维 修费用及保障费用 提 供 信 息,确 认 是 否 符 合 可 靠 性 定量要求 .我们通过试验一方面要获取可靠性数据, 另一方面要通过产 品 在 试 验 中 发 生 的 各 种 故 障,找 出其原因并进行细致的分析和研究以提高产品的可 靠性.但实际 机 械 零 件 的 设 计 方 案 如 果 有 改 变, 就必须重新进行一 次 试 验 分 析,这 需 要 花 费 很 大 的 人力、物力和财力,所以在可靠性试验前利用一些高 性能的软件进行模 拟 分 析 能 减 少 试 验 次 数,节 约 时 间和研究基金 。
	\item  传统设计方法与可靠性设计相结合;
	可靠 性 设 计 理 论 一 方 面 缺 少 基 础 数 据,另 一 方 面,即使对一个简单的结构,其失效模式也可能多到 难以计数,所以在一 般 产 品 的 设 计 中 推 行 可 靠 性 概 率的设计是十分困 难 的;传 统 安 全 系 数 法 虽 存 在 不 足,但有直观、简单、设计工作量小等优点,并在许多情况下能保证机械 零 件 的 可 靠 性,尤 其 对 不 重 要 的 情况或者因素非常 复 杂 且 难 精 确 分 析 的 情 况,有 丰 富经验基础的安全 系 数 法 很 有 价 值,因 此 还 不 能 完 全摒弃安全系数设 计 法 .目 前 采 用 概 率 设 计 法 的 概 念去完善和改进传 统 的 安 全 系 数,使 可 靠 性 和 安 全 系数直接联系,广泛 应 用 现 有 的 各 种 设 计 方 法 对 产 品进行设计计算,并 与 采 用 可 靠 性 概 率 设 计 方 法 得 出的结果以及实物 试 验 的 结 果 进 行 比 较,从 而 积 累 经验,收集和整理可靠性设计数据 。
\end{enumerate}

\section{结束语}

可靠 性 是 产 品 的 一 种 动 态 质 量 指 标,在 现 代 化 生产中已经贯穿在产品的开发、设计、制造、试验、使 用及维修保养的各个环节之中 ,所 以 可 靠 性 设 计 技术是机械工程学 科 重 要 的 研 究 方 向 之 一,而 就 机 械工程而言,其可靠 性 问 题 的 解 决 总 是 理 论 与 工 程 经验的结合,对于可靠性知识掌握越多,主观经验的 运用就会越少,自然机械结构的设计也就越合理,这 正是机械工程技术 研 究 追 求 的 目 标 .本文首先介绍机械零件可靠性的特点及其可靠性参数、 机械可靠性设计与传统机械设计之间的关系;结合可靠性理论对 机 械 可 靠 性 设 计 理 论 进 行 深 入 分 析。

\vspace{2em}
{ \noindent \sanhao \bf 参考文献:}
\begin{enumerate}[{[1]}]
	\item 张立博 . 探讨机械工程的可靠性优化设计 [J]. 科协论坛 : 下半 月 ,2012(03):38-39. 
	\item 班 德 利 . 电 器 产 品 的 可 靠 性 设 计 [J]. 城 市 建 设 理 论 研 究 ,2013(22):1-4. 
	\item 王志刚,戴柏林 . 可靠性技术的发展与应用 [M]. 北京:机械工 业出版社 ,2010.
	\item 尹廷亭 . 机械可靠性试验技术研究现状和展望 [J]. 科技创新导 报 ,2014(12):22-23.
	\item 杨家铿, 等. 可 靠性工程 [ Z ]. 广州: 电 子工 业质量 与可 靠性培训中心
	\item 李良巧, 等. 机械可靠 性设计 与分析 [M ]. 北京: 国防工 业出版社
\end{enumerate}

%%中文习惯是设定首行缩进为2em。注意此设置一定要在document环境之中,这可能与\setlength作用范围相关
%\setlength{\parindent}{2em}
%
%\title{XecjkTemplateTest}
%\author{XiaoHanyu}
%\maketitle
%
%\tableofcontents
%\listoffigures
%%\listoftablescontent
%
%\include{data/content}
%\include{data/appendix}
%
%%%加入参考文献支持
%\bibliography{data/main}
%%%解决目录中没有相应的参考文献的条目问题
%\addcontentsline{toc}{section}{\refname}
\end{document}
%%%%正文部分结束
%%%%%%%%------------------------------------------------------------------------
