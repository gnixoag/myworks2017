%%%% 导言区
%% 设定纸张大小为A4, 基本字体大小为12pt, 文章题目单独为一页, 
%% 文档类型为article
%\documentclass[12pt]{article}
%
%%% en_preamble包含基本的宏包配置
%%%%%%%%%------------------------------------------------------------------------
%%%% 日常所用宏包

%% 控制页边距
\usepackage[top=2cm, bottom=2cm, left=2.cm, right=2.cm,includehead,includefoot]{geometry}

%% 控制项目列表
\usepackage{enumerate}

%% 多栏显示
\usepackage{multicol}

%% hyperref宏包,生成可定位点击的超链接,并且会生成pdf书签
\usepackage[%
    pdfstartview=FitH,%
    CJKbookmarks=true,%
    bookmarks=true,%
    bookmarksnumbered=true,%
    bookmarksopen=true,%
    colorlinks=true,%
    citecolor=blue,%
    linkcolor=blue,%
    anchorcolor=green,%
    urlcolor=blue%
]{hyperref}

\ctexset {
	section = {
		name = {理论},
		format = \bf \zihao{4} \centering
	},
	subsection = {
		name = {},
		number = \Roman{subsection},
		format = \bf \zihao{4} 
	},
	subsubsection = {
		name = {,、\hspace{-0.5em}},
		number = \chinese{subsubsection},
		format = \bf \zihao{4} 
},
paragraph = {
name = {(,)\hspace{-0.5em}},
number = \chinese{paragraph},
format = \bf \zihao{4} 
},
subparagraph = {
name = {,、\hspace{-0.5em}},
format = \bf \zihao{4}
}
}

\setcounter{secnumdepth}{5}


%% 控制目录
\usepackage{titletoc}

%% 支持彩色文本、底色、文本框等
\usepackage{color,xcolor}

%%%% 基本插图方法
%% 图形宏包
\usepackage{graphicx}

%%%% pgf/tikz绘图宏包设置
\usepackage{pgf,tikz}
\usetikzlibrary{shapes,automata,snakes,backgrounds,arrows}
\usetikzlibrary{mindmap}

%%%% fancyhdr设置页眉页脚
%% 页眉页脚宏包
\usepackage{fancyhdr}

%% 页眉页脚风格
\pagestyle{plain}

%%%% 设置listings宏包用来粘贴源代码
%% 方便粘贴源代码,部分代码高亮功能
\usepackage{listings}

%% 所要粘贴代码的编程语言
%\lstloadlanguages{}

%% 设置listings宏包的一些全局样式
%% 参考http://hi.baidu.com/shawpinlee/blog/item/9ec431cbae28e41cbe09e6e4.html
\lstset{
showstringspaces=false,              %% 设定是否显示代码之间的空格符号
numbers=left,                        %% 在左边显示行号
numberstyle=\tiny,                   %% 设定行号字体的大小
basicstyle=\footnotesize,                    %% 设定字体大小\tiny, \small, \Large等等
keywordstyle=\color{blue!70}, commentstyle=\color{red!50!green!50!blue!50},
                                     %% 关键字高亮
frame=shadowbox,                     %% 给代码加框
rulesepcolor=\color{red!20!green!20!blue!20},
escapechar=`,                        %% 中文逃逸字符,用于中英混排
xleftmargin=2em,xrightmargin=2em, aboveskip=1em,
breaklines,                          %% 这条命令可以让LaTeX自动将长的代码行换行排版
extendedchars=false                  %% 这一条命令可以解决代码跨页时,章节标题,页眉等汉字不显示的问题
}
%%%% listings宏包设置结束

%% 设定段间距
\setlength{\parskip}{0.3\baselineskip}

%% 设定行距
\linespread{1}

\usepackage{tabu} % 用tabu代替 array
\usepackage{multirow}
\usepackage{zhnumber}
\usepackage{calc,marvosym,ifthen,fancybox,url,layout}
\setcounter{tocdepth}{1}
\usepackage{paralist}

%给旁注加个黑原点
\usepackage{wasysym}
\let\marginparNR\marginpar
\def\marginpar#1{\marginparNR{\textcolor{red}{ \CIRCLE{}   #1  }}}

%调整列表前后的间距
\makeatletter
\let\orig@Enumerate=\enumerate
\renewenvironment{enumerate}{\orig@Enumerate}{\vspace{-0.5cm}\endlist}
\let\orig@Itemize=\itemize
\renewenvironment{itemize}{\orig@Itemize}{\vspace{-0.5cm}\endlist}
\makeatother

%给目录进行设定
\titlecontents{section}[0pt]{\addvspace{5pt}\filright}
{ \thecontentslabel \hspace{0.5em} }
{}{\titlerule*[8pt]{.}\contentspage}


%画边框
%\def\boxhack{\leavevmode\vbox to0pt{\vss\rlap{\hskip 320pt
%			\setlength{\unitlength}{1pt}\cornersize*{10pt}\thicklines\fancyoval(365,675)}\vskip -680pt}}
%\def\boxhackb{\leavevmode\vbox to0pt{\vss\rlap{\hskip 80pt
%			\setlength{\unitlength}{1pt}\cornersize*{10pt}\thicklines\fancyoval(100,675)}\vskip -680pt}}

%用tikz画边框	
\def\biankuang{\leavevmode\vbox to0pt{
		\vss\rlap{\hskip 0.8cm
			\tikz \draw(4,0)--(0,0)--(0,-22.5)--(17,-22.5)--(17,0)--(4,0)--(4,-22.5);		
		}\vskip -22.7cm}}

\newcolumntype{M}[1]{>{\zihao{4}\centering\arraybackslash}m{#1}}
\newcolumntype{N}{@{}m{0pt}@{}}

\newcommand{\ktmq}[1]{\gdef\ktmqNR{#1}}%课题名称
\newcommand{\jxmb}[1]{\gdef\jxmbNR{#1}}%教学目标
\newcommand{\jxnd}[1]{\gdef\jxndNR{#1}}%教学难点
\newcommand{\jxzd}[1]{\gdef\jxzdNR{#1}}%教学重点
\newcommand{\jjff}[1]{\gdef\jjffNR{#1}}%解决方法
\newcommand{\jxhj}[1]{\gdef\jxhjNR{#1}}%教学后记

\newcommand{\jc}[1]{\gdef\jcNR{#1}}%教材
\newcommand{\cks}[1]{\gdef\cksNR{#1}}%参考书
\newcommand{\jsxm}[1]{\gdef\jsxmNR{#1}}%教师姓名
\newcommand{\jyszr}[1]{\gdef\jyszrNR{#1}}%教研室主任

\newcommand{\skbc}[1]{\gdef\skbcNR{#1}}%授课班次
\newcommand{\skrq}[1]{\gdef\skrqNR{#1}}%授课日期
\newcommand{\biaoti}[1]{\gdef\biaotiNR{#1}}%标题头

\newcounter{thesectionSY}

\newcommand{\makeshouye}{
	\setcounter{thesectionSY}{\thesection+1}
	\restoregeometry	
	\renewcommand{\headrulewidth}{0pt}
	\pagestyle{fancy}
	\fancyhead{}
	\lhead{} 
	\chead{
		\begin{tabular}{@{\hspace{1.2cm}}M{7cm}@{\hspace{-0.4cm}}M{8cm}N}
			\parbox{7cm}{\linespread{0.2}
				\makebox[7cm][s]{\kaishu \zihao{3} 湖南九嶷职业技术学院}\\ 
				\makebox[7cm][s]{\kaishu \zihao{3} 湖南潇湘技师学院}
			}
			&  \makebox[6cm][s]{\rule{0pt}{0.9cm}\zihao{1} \heiti \kaishu 授课课时计划}\\
		\end{tabular}
	}
	
	\begin{tabular}{M{2.2cm}|M{7cm}|M{5.8cm}N}
		\hline
		\multirow{2}*{
			\rule{0pt}{1.4cm}\parbox[b]{2.cm}{
				\centering 课\hfill 程\hfill 章\hfill 节\\及\hfill 主\hfill 题}}& \heiti \biaotiNR\thethesectionSY  &  ~授~课~教~师\hfill {\heiti \zihao{4} \underline{\jsxmNR}}\hfill 签字~~~&\\[0.6cm] \cline{2-3}
		
		& \heiti \ktmqNR &  ~教研室主任\hfill {\fangsong \bf \zihao{4} \underline{\jyszrNR}}\hfill 签字~~~&\\[0.6cm]\hline
		
		\multicolumn{3}{l}{
			\begin{minipage}[t][3.7cm][t]{15cm}	
				\begin{minipage}[t]{2.5cm}
					\vspace{6pt} \hfill \zihao{4} 教学目标:
				\end{minipage}\hspace{0.5cm}				
				\begin{minipage}[t][4.3cm][t]{12cm}
					\vspace{0pt}\zihao{4} \setlength{\baselineskip}{12pt} 
					\begin{enumerate}[1、]
						\jxmbNR
					\end{enumerate} 
				\end{minipage} 
			\end{minipage}
		}\vspace{0.3cm} &\\ \hline
		\multicolumn{3}{l}{
			\begin{minipage}[t][4cm][t]{15cm}
				\begin{minipage}[t]{2.5cm}
					\vspace{5pt} \hfill \zihao{4} 教学重点:
				\end{minipage}\hspace{0.5cm}				
				\begin{minipage}[t]{12cm}
					\vspace{0pt} \zihao{4} \setlength{\baselineskip}{12pt} 
					\begin{enumerate}[1、] \jxzdNR \end{enumerate}
					\vspace{7pt} 
				\end{minipage}
				\vspace{5pt} 
				\begin{minipage}[t]{2.5cm}
					\vspace{6pt} \hfill \zihao{4} 教学难点:
				\end{minipage}\hspace{0.5cm}		
				\begin{minipage}[t]{12cm}
					\vspace{0pt} \zihao{4} \setlength{\baselineskip}{12pt} 
					\begin{enumerate}[1、] \jxndNR \end{enumerate}
					\vspace{0pt} 	
				\end{minipage}
				\begin{minipage}[t]{2.5cm}
					\vspace{6pt} \hfill \zihao{4} 解决方法:
				\end{minipage}\hspace{0.5cm}		
				\begin{minipage}[t]{12cm}
					\vspace{6pt}\zihao{4} \jjffNR
				\end{minipage}
				
			\end{minipage}
		} &\\  \hline
		
		\multirow{2}*{ 	\rule{0pt}{1.4cm}\parbox[b]{2.cm}{
				\centering 教\hfill 材\hfill 和\\参\hfill 考\hfill 书 } } & \multicolumn{2}{c}{\zihao{4} \jcNR } &\\[0.6cm] \cline{2-3}
		&  \multicolumn{2}{c}{\zihao{4} \cksNR } &\\[0.6cm] \hline
		\multirow{2}*{\rule{0pt}{1.4cm}\parbox[b]{2.cm}{
				\centering 授\hfill 课\hfill 班\hfill 次\\授\hfill 课\hfill 日\hfill 期 } } & \multicolumn{2}{c}{ \skbcNR } &\\[0.6cm] \cline{2-3}
		&\multicolumn{2}{c}{ \skrqNR } &\\[0.6cm] \hline
		
		\multicolumn{3}{l}{
			\begin{minipage}[t]{2.5cm}
				\vspace{0pt} \hfill \zihao{4} 教学后记:
			\end{minipage}\hspace{0.5cm}				
			\begin{minipage}[t][4.5cm][t]{12cm}
				\vspace{0pt}\zihao{4} \jxhjNR
			\end{minipage} 
		}\vspace{0.3cm} &\\ \hline	
	\end{tabular}
	\newpage
	\newgeometry{textwidth={\textwidth-150pt},top=2cm,bottom=2cm,right=2.5cm,includehead,includefoot,marginparsep=28pt,marginparwidth=85pt}
	
	\reversemarginpar
	\fancyhead{} 
	\chead{\hspace{1cm} \kaishu \zihao{1} 教 \hspace{1cm} 案 \hspace{1cm} 纸 }
	%\lhead{\boxhack \boxhackb } %边框 
	\lhead{ \biankuang}%边框 
	\zihao{4}		

	\section{ \ktmqNR }
}


%
%%% 如果不写中文的话就不需要引用xecjk_preamble里面的配置
%\input{data1/xecjk_preamble}

%\usepackage{tabu} % 用tabu代替 array
%\usepackage{multirow}
%\usepackage{zhnumber}
%\usepackage{calc,marvosym,ifthen,fancybox,url,layout}
{
\clearpage
\eject \pdfpagewidth=370mm \pdfpageheight=260mm
\newgeometry{top=2cm, bottom=2cm, left=2.cm, right=-14.cm,includefoot}

\newcolumntype{M}[1]{>{\sihao\centering\arraybackslash}m{#1}}
\newcolumntype{N}{@{}m{0pt}@{}}

\setlength{\parindent}{0pt}

%% 首页格式
%\input{data/shouye}

%%%% 导言区结束
%%%%%%%%------------------------------------------------------------------------

%%%%%%%%------------------------------------------------------------------------

%%%% 正文部分
%\begin{document}
\setlength{\columnsep}{30pt } 
\columnseprule=0pt \twocolumn

\begin{center}
 \begin{tabular}{M{12em}M{4cm}N}
	\parbox{12em}{ \linespread{0.2}\zihao{-4} \bf \songti	
		\makebox[12em][s]{湖南九嶷职业技术学院}\\[0.1cm]
		\makebox[12em][s]{湖南潇湘技师学院}
	}
	&  \zihao{1} \heiti \makebox[4cm][s]{\rule{0pt}{0.9cm}授课计划}&\\
\end{tabular}
\vspace{0.8cm}

\zihao{3} \bf  \underline{~~~2016--2017~~~} 学年  \underline{~~~2~~~}  学期
\end{center}

 \begin{tabular}{Nllll}
&系部:\underline{\makebox[12em]{\textbf{机电工程系}}}& 
专业: \underline{\makebox[7em]{\textbf{数控技术}}}&
班级: \underline{\makebox[9em]{\textbf{15级中数班}}}& \\[2ex]
&课程:\underline{\makebox[12em]{\textbf{《数控编程与实习》}}}&
上课周数:\underline{\makebox[5em]{\textbf{17}}} &
周学时:  \underline{\makebox[8em]{\textbf{2[3](3)}}} & 
\end{tabular}
\vspace{0.3ex}

 {\bf 本学期课时分配表}
\vspace{0.ex}

%\hspace{-0.5em}
\begin{tabular}{|M{4em}|M{2.5em}|M{2.5em}|M{2.5em}|M{2.5em}|M{2.5em}|M{2.5em}|M{2.5em}|M{2.5em}|M{3em}|N}
	\hline 
	教学\par 模式 & \multicolumn{2}{c|}{理论}& \multicolumn{2}{c|}{一体化} & \multicolumn{2}{c|}{实习}&  \multirow{3}{*}{考}& \multirow{3}{*}{机}  &
	 \multirow{3}{*}{合} & \\[4.5ex]
	\cline{1-7} 
	教学 \par  形式& 讲\par  课 & 实\par  验 & 理\par 论\par 讲\par 课& 实\vspace{5.5ex} 训 & 理\par 论\par 讲\par 课 & 生\par 产\par 实\par 习 &\multirow{3}{*}{核}&\multirow{3}{*}{动}&\multirow{3}{*}{计}&\\ [12ex]
	\hline 
	 课时&×&×&30&\wuhao [48](48)&×&×&2&\wuhao 2[3](3)&\wuhao 34[51](51)& \\[4ex]
	\hline 
\end{tabular} 

~\vspace{0.3ex}

 说明:与本课程无关教学模式的各项各打×
\vspace{0.5ex}

 备注:~~
\begin{minipage}[t]{15cm}\vspace{-1.25em}
\begin{enumerate}[1、]
	\item 本课程以前完成学时数:\underline{\makebox[22em]{\textbf{64[48]{48}}}}
	\item 本课程在以后学期尚余留时数:\underline{\makebox[19em]{\textbf{[64]{48}  }}}        
	\item 本课程本学期列为考试(考查)课程:\underline{\makebox[16.8em]{\textbf{理论考试(实习考查)  }}} 
	\item 本课程使用教材名称: \underline{\makebox[23em]{\textbf{数控机床编程与操作(数控铣床~加工中心分册)}}}
\end{enumerate}
\end{minipage}
\vspace{0.5ex}

\newcommand{\ud}[2]{\underline{\makebox[#1]{\textbf{#2}}} }
\setlength{\baselineskip}{1.5\baselineskip}
\makebox[5em][s]{任课教师}:\ud{10em}{}  \hspace{1em} 编写日期:\ud{5.5em}{}年\ud{3em}{}月\ud{3em}{}日\\
\makebox[5em][s]{教研室主任}:\ud{10em}{}  \hspace{1em} 编写日期:\ud{5.5em}{}年\ud{3em}{}月\ud{3em}{}日\\
\makebox[5em][s]{系主任}:\ud{10em}{}  \hspace{1em} 编写日期:\ud{5.5em}{}年\ud{3em}{}月\ud{3em}{}日\\
\makebox[5em][s]{教务处}:\ud{10em}{}  \hspace{1em} 编写日期:\ud{5.5em}{}年\ud{3em}{}月\ud{3em}{}日\\
\makebox[5em][s]{分管领导}:\ud{10em}{}  \hspace{1em} 编写日期:\ud{5.5em}{}年\ud{3em}{}月\ud{3em}{}日\\

\pagebreak

\begin{center}
\erhao \hei 学期授课计划说明
\end{center}
\xiaosi \setlength{\parindent}{2em} \setlength{\baselineskip}{22pt}

\textbf{一、教学目的与要求:}

本学期主要在上个学期的基础上学习数控编程中的手工编程,要求学生能熟练运用各种编程方法来解决实际问题,充分把自己的能力及智慧通过编程展示出来。为以后走上工作岗位作好准备。

\textbf{二、用教材、参考书}

1、使用教材: 《数控机床编程与操作(数控铣床 加工中心分册)》 沈建峰

2、参考书:《加工中心编程与操作》  科学出版社  刘加孝   主编

\hspace{4.5em}《加工中心操作工》 中国劳动社会保障出版社  杨伟群  主编

\hspace{4.5em}《加工中心考工实训教程》  化学工业出版社   吴明友 主编

\textbf{三、教学措施}

1、采用多媒体、仿真、讨论等教学方法。

2、作业:理论课每周布置一道编程题,仿真每周做习题集上的题目,实习除了完成课题外,还要每个课题写一个实习报告。

3、学生评价采用自评、小组评价、教师评价三结合。

\textbf{四、增删内容}

本计划无增删内容。

\textbf{五、本课程与其他课程的关系}

本课程是专业课,其他课程是基础,为本课服务。先要学习好《数控加工工艺》、《普铣》、《机械制图》、《机械加工原理》、《专业数学》等课程。在这些课程的基础上再来学习本课程就容易多了,希望同学们多复习这些课程。

\textbf{六、课程计划周数:}

授课时间为2--18周(第1周学生生报到注册,第19周考试),周课时8节。

\newpage 
\onecolumn \setlength{\parindent}{0em}

\begin{center}
	\begin{tabular}{M{12em}M{4cm}N}
		\parbox{12em}{\linespread{0.2} 
			\xiaosi \bf \song	\makebox[12em][s]{湖南九嶷职业技术学院}\\[0.1cm]
			\makebox[12em][s]{湖南潇湘技师学院}
		}
		&  \yihao \hei \makebox[8cm][s]{\rule{0pt}{0.9cm}教师学期授课计划}&\\
	\end{tabular}
\end{center}
 
\newcolumntype{M}[1]{>{\xiaosan\arraybackslash}m{#1}}
\begin{tabular}{|>{\centering}M{1.5cm}|M{6cm}|M{9cm}|>{\centering}M{4cm}|>{\centering}M{5cm}|>{\centering}M{2cm}|>{\centering}M{2.5cm}|N}
 	\hline 
 周次&\centering 授课章节内容摘要&\centering 教学 要求& 教具及实验\par 实习材料& 作业及参考材料& 教学\par 时数& 备注& \\[4.5ex] \hline
1& 教师报到、学生报到 变量编程概述		& & & & & 02.13 02.19 & \\[4.5ex] \hline
2& 理论1、复习导入 & 复习上学期所学内容&自绘示意图1&习题1& 2节 & 02.20 02.26 &  \\[4.5ex] \hline
3& 理论2、变量编程概述 &掌握变量及用变量来编程 &自绘示意图2&习题2 & 2节 & 02.27 03.05 & \\[4.5ex] \hline
4& 理论3、变量Z向分层 &掌握Z向分层的应用 &自绘示意图3&习题3 &2节 & 03.06 03.12 & \\[4.5ex] \hline
5& 理论4、椭圆编程	&掌握椭圆加工及while指令 &自绘示意图4&习题4 & 2节& 03.13 03.19 & \\[4.5ex] \hline
6& 理论5、椭圆弧编程 	& 掌握椭圆弧的加工&自绘示意图5 &习题5 &2节 & 03.20 03.26 & \\[4.5ex] \hline
7& 理论6、局部坐标系 & 掌握局部坐标系的使用& 自绘示意图6&习题6 & 2节& 03.27 04.02 & \\[4.5ex] \hline
8& 理论7、坐标系旋转(一) &掌握坐标系旋转的使用 & 自绘示意图7&习题7 & 2节& 04.03 04.09 & \\[4.5ex] \hline
9& 理论8、坐标系旋转(二)& 掌握坐标系旋转的编程&自绘示意图8 &习题8 &2节& 04.10 04.16 & \\[4.5ex] \hline
10& 理论9、极坐标指令 & 掌握极坐标指令的使用& 自绘示意图9& 习题9& 2节& 04.17 04.23& \\[4.5ex] \hline
 \end{tabular} 

\newpage 
\begin{center}
	\begin{tabular}{M{12em}M{4cm}N}
		\parbox{12em}{\linespread{0.2}
			\xiaosi \bf \song	\makebox[12em][s]{湖南九嶷职业技术学院}\\[0.1cm]
			\makebox[12em][s]{湖南潇湘技师学院}
		}
		&  \yihao \hei \makebox[8cm][s]{\rule{0pt}{0.9cm}教师学期授课计划}&\\
	\end{tabular}
\end{center}

\begin{tabular}{|>{\centering}M{1.5cm}|M{6cm}|M{9cm}|>{\centering}M{4cm}|>{\centering}M{5cm}|>{\centering}M{2cm}|>{\centering}M{2.5cm}|N}
	\hline 
	周次&\centering 授课章节内容摘要&\centering 教学 要求& 教具及实验\par 实习材料& 作业及参考材料& 教学\par 时数& 备注& \\[4.5ex] \hline
	11& 理论10、期中测试 	& 期中测试&自绘示意图10 &习题10 &2节 & 04.24 04.30& \\[4.5ex] \hline
	12& 五一放假 机动		 & & & & & 05.01 05.07& \\[4.5ex] \hline
	13& 理论11、试卷讲解 &复习巩固& 自绘示意图11&习题11 &2节 & 05.08 05.14& \\[4.5ex] \hline
	14& 理论12、孔系变量编程&掌握孔系变量编程技巧& 自绘示意图12&习题12& 2节& 05.15 05.21& \\[4.5ex] \hline
	15& 理论13、变量周边导圆角 &掌握变量周边导圆角编程技巧 & 自绘示意图13&习题13 & 2节& 05.22 05.28& \\[4.5ex] \hline
	16& 理论14、自动编程 & 掌握自动编程的流程& 自绘示意图14&习题14 &2节 & 05.29 06.04& \\[4.5ex] \hline
	17& 理论15、综合练习 & 了解自动编程的技巧&自绘示意图15 &习题15 & 2节& 06.06 06.11& \\[4.5ex] \hline
	18& 理论16、期末复习 & 复习& 自绘示意图16&习题16& 2节& 06.12 06.18& \\[4.5ex] \hline
	19& 期末考试、阅卷 & & & & & 06.19 06.25& \\[4.5ex] \hline
	&  			 & & & & & & \\[4.5ex] \hline
\end{tabular} 
\vspace{1ex}

\hspace{10cm}  {\sanhao    任课教师:\ud{8em}{} \hfill 教研室主任:\ud{8em}{}  \hfill 系主任: \ud{8em}{}  \hfill }


\newpage 
\begin{center}
	\begin{tabular}{M{12em}M{4cm}N}
		\parbox{12em}{\linespread{0.2}
			\xiaosi \bf \song	\makebox[12em][s]{湖南九嶷职业技术学院}\\[0.1cm]
			\makebox[12em][s]{湖南潇湘技师学院}
		}
		&  \yihao \hei \makebox[8cm][s]{\rule{0pt}{0.9cm}教师学期授课计划}&\\
	\end{tabular}
\end{center}

\begin{tabular}{|>{\centering}M{1.7cm}|M{6cm}|M{9cm}|>{\centering}M{4cm}|>{\centering}M{4.5cm}|>{\centering}M{2.5cm}|>{\centering}M{2.3cm}|N}
	\hline 
	周次&\centering 授课章节内容摘要&\centering 教学 要求& 教具及实验\par 实习材料& 作业及参考材料& 教学\par 时数& 备注& \\[4.5ex] \hline
	1& 学生报到注册 	& & & & & 02.20 02.26& \\[4.5ex] \hline
	2-4& 实习1、六面四方体加工 &掌握平面的加工\par 掌握六面四方体的加工工艺 &数控机床及\par 相关工具 &实习报告1 & [9](9)& 02.27 03.12& \\[4.5ex] \hline
	5-8& 实习2、六面圆槽加工 &掌握槽的下刀方式\par 掌握六面圆槽的加工工艺 &数控机床及\par 相关工具 & 实习报告2& [12](12)& 03.13 04.09& \\[4.5ex] \hline
	9-11& 实习3、椭圆加工 &掌握椭圆的宏程序\par 掌握椭圆的加工工艺 &数控机床及\par 相关工具 &实习报告3 &  [9](9) & 04.10 04.30& \\[4.5ex] \hline
	12& 五一放假 & & & & & 05.01 05.07& \\[4.5ex] \hline
	13-17& 实习4、薄壁配合加工 &掌握薄壁的加工工艺\par 掌握配合件的加工工艺&数控机床及\par 相关工具 &实习报告4 &  [15](15)& 05.08 06.11& \\[4.5ex] \hline
	18& 复习 &复习总结 & & & &06.12 06.18 & \\[4.5ex] \hline
	19& 期末考试、阅卷 & & & & &06.19 06.25 & \\[4.5ex] \hline
	&  & & & & & & \\[4.5ex] \hline
	& & & & & & & \\[4.5ex] \hline
\end{tabular} 
\vspace{1ex}

\hspace{10cm}  {\sanhao    任课教师:\ud{8em}{} \hfill 教研室主任:\ud{8em}{}  \hfill 系主任: \ud{8em}{}  \hfill}


\eject \pdfpagewidth=210mm \pdfpageheight=297mm

}
%% 中文习惯是设定首行缩进为2em。注意此设置一定要在document环境之中,这可能与\setlength作用范围相关
%\setlength{\parindent}{2em}                    
%
%\title{Xecjk Template Test}
%\author{Xiao Hanyu}
%\maketitle
%
%\tableofcontents
%\listoffigures
%%\listoftablescontent
%
%\include{data/content}
%\include{data/appendix}
%
%%% 加入参考文献支持
%\bibliography{data/main}
%%% 解决目录中没有相应的参考文献的条目问题
%\addcontentsline{toc}{section}{\refname} 
%\end{document}
%%%% 正文部分结束
%%%%%%%%------------------------------------------------------------------------
