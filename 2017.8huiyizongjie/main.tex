\documentclass[UTF8,zihao=-4]{ctexart}
\usepackage[a4paper,top=2.cm, bottom=2.cm, left=3.18cm, right=3.18cm,includehead,includefoot]{geometry}
\usepackage{fancyhdr}
\usepackage{enumerate}
\pagestyle{plain}
\usepackage[%
pdfstartview=FitH,%
CJKbookmarks=true,%
bookmarks=true,%
bookmarksnumbered=true,%
bookmarksopen=true,%
colorlinks=true,%
citecolor=blue,%
linkcolor=blue,%
anchorcolor=green,%
urlcolor=blue%
]{hyperref}

\ctexset {
	section = {	name = {,、\hspace{-1em}},number = \chinese{section},
		format = \bf \zihao{4}	},
	subsection = {
		name = {\hspace{1.5em}(,)\hspace{-1em} },
		number = \chinese{subsection},
		format = \zihao{4} },
	subsubsection = {
		name = {\hspace{2em},、\hspace{-1em}},
		number = \arabic{subsubsection},
		format =  \zihao{4}}
}


\title{湖南省高职装备制造类专家委员会\\“制造强省”高峰论坛总结}
\author{高~~星}
\date{}

\begin{document}

\begin{center}
\zihao{3} \heiti	湖南省高职装备制造类专家委员会\\“制造强省”高峰论坛总结

    湖南九嶷职业技术学院~~~~高~~星
\end{center}

\section{会议基本情况}
本次“制造强省”高峰论坛于8月26日在中共湖南省高等学校工作委员会党校开始。本次会议讨论了省高职装备专业委员会工作计划,解读了“中国制造2025”以及长株潭衡“中国制造2025”试点示范城市群建设推进计划,每个学院还发言讨论了高职院校如何为长株潭衡“中国制造2025”试点示范城市群建设进行服务,通过本次会议,我增长了见识,扩充了视野,收获颇丰。
\section{会议内容}
\subsection{湖南省高职装备制造类专业委员会工作总结}
\subsubsection{加强支柱产业调研}
2016年5-6月,组织专题调研组对长株潭地区重点装备制造企业进行了新一轮的调研,实地调研了中联重科股份有限公司、湘电集团有限公司、中车株洲电力机车有限公司、广汽三菱汽车有限公司、湖南猎豹汽车股份有限公司等15家重点企业,问卷调查了100家骨干企业,形成了湖南装备制造业人才需求调研报告。

\subsubsection{深化重点企业合作,}
专委员积极牵线搭台,促进学校与企业合作,工业职院、机电学院,电气职院,信息职院等学校与博世,中联重科,维德科技,九城集团,吉利汽车,上海大众,华曙高科等企业续签校企合作协议,与亚信科技、华中数控、铁建重工、湖南航天磁电、屹丰模具、湖南高至科技、湖南五新隧道等企业建立合作关系,并为下一步人才培训、科研合作、共建大师工作室、混合制企业学院,打下了坚实基础。

\subsubsection{丰富开放办学成果}
目前,我省正在大力实施开放崛起战略,为追求卓越,创建一流高职,服务,湖南装备制造业走出去战略,相关学校开设海外订单班有“中联班”、“博世班”、“地铁班”、“大华班”、“吉利班”、“九城班”、“长丰班”、“长九博腾班”、“亚信班”等,学生人数达1000余人,去年,专委会学校委员国际合作办学实现零的突破,湖南工业职院与加拿大北方应用理工学院开设的机械制造专业合作班正式开班了,同时专委员会积极参与美国、德国、瑞士等高校开展了国际交流活动。
\subsubsection{搭建对接合作平台}
专委会同机械装备职教集团组织以“弘扬工匠精神,推动制造强省”为主题的技能大师进校园活动,全省装备制造业重点企业21名国家及省级劳模型工匠大师与专委会内的学院教师进行了对接,弘扬工匠精神,进一步深化产教融合,引起了社会强烈反响。组织2016湖南省机械装备制造类学校企业人才对接活动以及校企合作座谈会,探讨校企合作机制的创新,进行问卷调查,形成了企业人才需求调研报告。
\subsection{解读湖南省高职装备制造类专业委员会工作计划}
\subsubsection{工作计划确定了指导思想}
以“服务湖南制造强省”为目标,以“推动一流专业建设、培养一流技术技能人才
”为核心,围绕“服务、协调、维权”三大职能,树立“服务政府、服务学校、服务教师、服务企业”的理念,坚持产教融合,提升服务产业发展的能力,引领职业教育发展;协调好政府与学校、企业之间的关系,协调爱完善技术技能人才的三个对接培养体制。推动职业教育专业设置与实现产业需求对接,教学课程内容与职业人才需求标准对接,教学实训过程与生产制造过程对接;维护会员单位和个人参与专委会管理、公平获取学会公共资源的合法权益,促进广大会员自我完善、自我规范,满足专业教学科研和管理的需求,加强管理能力,提高专委会的工作质量、提升工作水平,努力将为专委会建成“为教育行政决策服务、为院校可持续发展服务、为教师专业化成长服务”的符合型社会专业组织。
\subsubsection{工作计划确定了发展目标}
瞄准湖南省制造强省的五年行动计划,着力推进我省高职教育服务湖南装备制造业的能力、推进委员会的建设,推动政府、企业和高职院校之间的交流,进一步提高专委会和行业的凝聚力,指引力和影响力。
\begin{enumerate}[\hspace{2.5em}A、]
	\item 整合三方力量,提升协调能力;
	\item 加强产教融合,提高服务能力;
	\item 实现三个对接,充分发挥作用。
\end{enumerate}
\subsubsection{工作计划确定了主要任务}
\begin{enumerate}[\hspace{2.5em}A、]
	\item 加强组织建设,逐步壮大专委会的力量;
	\item 加强组织协调,实现共同发展;
	\item 加强组织引导,实现三个对接,
	\item 加强组织指导,推动一流专业群的建设。
\end{enumerate}
\subsubsection{工作计划确定了保障措施}
按照委员会三年计划的目标、任务和要求,确定每年的工作重点,制定年度工作计划,细化实施步骤,明确责任人,有组织、有计划推荐各项工作等。

\subsection{解读《长株潭衡“中国制造2025”试点示范城市群建设推进计划(2017-2019)》}
长株潭衡城市群是长江中游城市群的核心之一,是国家两型社会综合配套改革试验区,是湖南制造强省建设的主战场。近年来获得了“两型社会”、自主创新、两化融合等多个国家级试点示范。具有较好的产业基础和体制机制优势。

立足长沙,株洲、湘潭、衡阳四市产业基础和优势,分业施策,有序推进,大力发展12大重点产业,构筑长株潭衡城市群高端、智能、高效、绿色的新型制造体系。

长沙市大力实施智能制造工程,围绕智能装备与产品,智能生产与服务等关键重点领域,支持产证学研用联合攻关,开发智能产品和自主可控的智能装置并实现产业化,依托优势企业建设智能工厂和数字化车间。大力发展工程机械高端整机产品和核心零件,提高核心技术自主,打造技术智能化、制造服务化、服务网络化、资源配置全球化的国际领先的工程机械之都。

株洲市大力实施制造业创新能力建设工程,围绕制造业创新发展的重大共性需求,形成以国家级创新中心为代表的重大创新平台,开展行业基础和共性技术研发,成果产业化、专利转化、人才培训等。突破先进轨道交通装备行业关键共性技术,促进轨道交通装备产业升级,优化产业空间布局,以龙头企业为牵引加快打造世界领先的现代化轨道交通装备制造业产业体系。

衡阳市大力实施高端装备创新工程,围绕高端制造装备共性关键技术突破与工程化、产业化瓶颈,建设一批重点工程项目,组织一批重点攻关,开发一批标志性、带动性强的重点产品和重大装备,提升自主设计水平和系统集成能力。在海洋工程装备领域,对接“国家海洋强国”和“一带一路”战略,以龙头企业为支撑,引入智能化生产技术,打造智能化工厂,形成海洋工程装备产业链,打造国内领先的海洋工程装备生产基地。

衡阳市大力发展实施工业强基工程。做强电力装备、新一代信息技术等产业,围绕核燃料研、核设备制造、核尾矿库综合治理等核产业开发,建设衡阳白沙绿岛军民融合产业园、促进军民融合深度发展,加快建设军民融合产业发展试点基地。

长沙、株洲、湘潭,衡阳四点作为试点示范的主体,要强化对试点示范工作的组织领导和协调推进,健全工作机制,落实责任分工,制定推进工作计划,完善配套支持政策,加强监督检查,扎实抓好各项试点任务落实一个。加快形成长沙“麓谷”、株洲“动力谷”、湘潭“智造谷”,衡阳先进制造业基地协同错位特色发展的格局。

\subsection{分组讨论} 
分组讨论我们主要讨论“如何服务湖南制造强省五年行动计划和长株潭衡中国制造2025试点示范城市群建设以及专业委员会工作计划。讨论会上每个学院都进行了发言,把各自学院目前的状态进行了讲解,分享了各专业建设的经验以及后发展的思路与理念,提出了现在存在的问题,大家都讲到了很多,对专业委员会工作计划以及如何服务湖南制造强省五年工作计划和长株潭衡中国制造2025试点示范城市群建设提供了宝贵意见。

\section{感想及建议}
此次会议开展的非常成功,让我学到了很多,感慨万分。

首先是每个学校都积极探索校企合作,哪个学校的校企合作搞得好,那他这个学校就搞得好,很多学校都探索出了自己的一条路。我们学校也要支持探索校企合作,借鉴其他学校的经验努力。

其次专业建设要服务地方、差异竞争、错位发展、局部突破。要实时分析国家的政策、省里的方针,以及本市的趋势,更好的培养出优秀人才。

很多学校对数控模具专业进行了专业调整和整合,加入了智能制造,如工业职院把机器人技术开成了一门公共课。有的学校把机器人专业、3d打印增材制造专业、数控模具智能制造、航空飞行器维修专业等打造成了一个智能制造的专业群。我们学校也需要进行调整。

校企合作要找到一个突破口,如岳阳职院,他在一次机遇下加入了中国游乐同盟,马上就把游乐制造做得非常好,形成了自己的特色。民政职院,进行了错位发展,开设了助老服务机器人等。

装备制造业委员会,把企业、学校、政府三方联合在一起,可以集中力量干大事,但我们学校并不是它的成员,他们的很多成果我们就分享不到,比如这次的调研报告等,希望我们也能加入进来、抱团取暖。

这次的收获非常大,非常感谢学校对我的关心和支持,下次有机会我还想去,希望能把这次学到的东西消化掉,运用到今后的工作当中去,为学校的发展做出自己的贡献。


\flushright  2017年9月5日 \hspace{2cm}

\end{document}